\documentclass{article}
\usepackage{amsthm}
% \usepackage{ntheorem}
\usepackage{amsfonts}
\usepackage{graphicx}

% define break theorem style
\newtheoremstyle{break}
  {\topsep}{\topsep}%
  % {\itshape}{}%
  {\normalfont}{}%
  {\bfseries}{}%
  {\newline}{}%

% define definition 
\theoremstyle{break}
\newtheorem{theorem}{Theorem}
\newtheorem*{theorem*}{Theorem}
\newtheorem*{corollary*}{Corollary}

\newcommand*{\Z}{\mathbb{Z}}
\newcommand*{\N}{\mathbb{N}}

\begin{document}
\title{Select Theorems in Abstract Algebra}
\author{Riley Weber}
\maketitle

Taken from \underline{Abstract Algebra: An Introduction} by Thomas W.
Hungerford (ISBN 978-1111569624). Created to study while taking MATH 3163:
Modern Algebra at UNC Charlotte. Theorems are based on what was covered in
class. Theorems are ordered as they appear in the book and are sectioned by
chapter.

\section*{Chapter 4}
% optional, to include this section in the toc
% \addcontentsline{toc}{section}{Chapter 4}
\begin{theorem*}[4.2]
  If $R$ is an integral domain and $f(x), g(x)$ are non-zero polynomials in
  $R[x]$, then deg$[f(x)g(x)] = $ deg$[f(x)] +$ deg$[g(x)]$
\end{theorem*}

\begin{corollary*}[4.3]
  If $R$ is an integral domain, the so is $R[x]$
\end{corollary*}

\begin{corollary*}[4.4]
  Let $R$ be a ring. If $f(x)$, $g(x)$, and $f(x)g(x)$ are nonzero in $R[x]$,
  then deg$(f \cdot g) \leq $ deg$(f) + $ deg $(g)$
\end{corollary*}

\begin{corollary*}[4.5]
  Let $R$ be a an integral domain and $f(x) \in R[x]$. Then:
  \begin{itemize}
    \item $f(x)$ is a unit in $R[x]$ iff $f(x)$ is a constant polynomial that
    is a unit in $R$
    \item In particular, if $R$ is also a field, then the units in $R[x]$ are
    the nonzero constants in $F$
  \end{itemize}
\end{corollary*}

\begin{theorem*}[4.6 The Division Algorithm for Polynomials]
  Let $F$ be a field and $f(x), g(x) \in F[x]$ with $g(x) \neq 0_F$. Then,
  there exist unique polynomials $q(x)$ and $r(x)$ such that: \\
  $f(x) = g(x)q(x) + r(x)$ and either $r(x) = 0_F$ or deg $r(x) <$ deg $g(x)$. 
\end{theorem*}

\begin{corollary*}[4.9]
\end{corollary*}

% \begin{theorem*}[]
% \end{theorem*}

\end{document}
