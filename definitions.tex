\documentclass{article}
\usepackage{amsthm}
\usepackage{amsfonts}

\theoremstyle{definition}
\newtheorem{definition}{Definition}[subsection]

\newcommand*{\Z}{\mathbb{Z}}

\begin{document}
\title{Definitions for Abstract Algebra}
\author{Riley Weber}
\maketitle

Taken from Abstract Algebra: An Introduction by Thomas W. Hungerford (ISBN
978-1111569624). Created to study while taking MATH 3163: Modern Algebra at UNC
Charlotte. Definitions orderd as they are in the book.

\section{Chapter 1}
\subsection{Section 1.1}

\begin{definition}[Well-Ordering Axiom] 
  every non-empty subset of the set of non-negative integers has a least 
  element
\end{definition}

\subsection{Section 1.2}
\begin{definition}[Divisibility]
  Let $a, b \in \Z$ with $b \neq 0$. We say that $b$ divides $a$ and
  write $b \mid a$ if $a=bc$ for some $c \in \Z$.
\end{definition}

\begin{definition}[Greatest Common Divisor]
  Let $a, b \in \Z$, not both zero. The greatest common divisor ($gcd$) is
  the greatest integer that divides both $a$ and $b$. This means that if $d$ is
  the $gcd$ of $a$ and $b$, then
  \begin{enumerate}
    \item $d \mid a$ and $d \mid b$
    \item if $c \mid a$ and $c \mid b$, then $c \leq d$
  \end{enumerate}
  The greatest common divisor is often written $d = gcd(a,b)$ or simply
  $(a,b)$. it is also frequently called the greatest common \emph{denominator}.
\end{definition}

\begin{definition}[]
\end{definition}

\begin{definition}[]
\end{definition}

\end{document}
